%!TEX TS-program = xelatex
%\documentclass[]{resume}
\documentclass[]{resume}
\usepackage{graphicx}

\begin{document}

\graphicspath{ {images/} }

% --- Header --- %
\header{Michael }{Pobega}
       {Linux Systems Engineer}

% --- Footer --- %
\footer{References, Capstone documentation and personal projects available on my Github}

% --- Left Column --- %
\begin{aside}
%  \includegraphics{caricaturesmall}
  \section{Contact}
    pobega@gmail.com
    (917) 436 - 0950
  \section{Skills}
    Linux system admin, 
    Kernel debugging,
    Application debugging,
    Shell scripting, Automation, Programming
  \section{Languages}
    Python, Rust, Perl,
    Javascript, Bash, C/C++
  \section{Systems}
    ChromiumOS, Gentoo, Fedora/Silverblue, Debian, RedHat, FreeBSD
  \section{Tools}
	Ansible, Nagios, Flatpak, Toolbox, Docker/Podman, nmap, Git, gdb, MongoDB, Virsh, Qemu, repo, Jenkins, Glade (GTK)
  \section{Online}
    {\bodyfontbold pobega}.github.io
    github.com/{\bodyfontbold pobega}
    linkedin.com/in/{\bodyfontbold pobega}
    serverfault.com/{\bodyfontbold pobega}
\end{aside}

% --- Right Column --- %
\begin{main}
  \section{Work Experience}
    \experience{Neverware}{New York, NY}{Software Developer}{September 2016 - Present}
      \point {Developer on Neverware's fork of ChromiumOS  "CloudReady"}
        \subpoint {Primarily focused on hardware compatibility and regression fixes}
      \point {Debugs and resolves hardware compatibility issues in CloudReady}
        \subpoint {Collaborates with Product and Sales to resolve customer issues}
        \subpoint {Manages driver compatibility and patches hardware quirks in the Linux kernel}
      \point {Merges upstream code back into Neverware's forks and resolves conflicts and build issues}
      \point {Maintains a Python tool to automate the creation of deployable CloudReady installer images}
        \subpoint {Saves Neverware's support team 2 hours per customer request}
    \experience{Neverware}{New York, NY}{Site Reliability Engineer}{September 2014 - September 2016}
      \point {Administrator for Neverware's local Windows 7 Hypervisor solution "PCReady"}
        \subpoint {Managed roughly 300 PCReady servers across 120 schools}
      \point {Worked closely with the Engineering team to find and fix core product issues}
      \point {Engineered software in Python for monitoring and automation of PCReady tasks}
      \point {Handled customer support interactions for PCReady via phone and e-mail}
    \experience{SUNYIT ITS Department}{Utica, NY}{Linux System Administrator}{September 2013 - May 2014}
      \point {Migrated services from legacy Unix hardware to a modern virtualized environment}
      \point {Reimplemented legacy mailing list software in Perl and integrate with existing LDAP}
      \point {Introduced configuration management through Ansible as a tool for rapid redeployment}
      \point {Setup Nagios server monitoring for the school's production network}
    \experience{SUNYIT CS Department}{Utica, NY}{Junior System Administrator}{January 2010 - January 2012}
      \point {Responsible for the Gentoo Linux lab, as well as other misc. Linux labs on campus}
        \subpoint {Lab was managed via PXE boot multicast imaging process.}
      \point {Implemented multicast ISO burning solution in Bash (\textit{wodimcast})}
  \hrulefill
  \section{Education}
    \experience{SUNY Institute of Technology}{Utica, NY}{BSc Computer Science}{Class of May 2014}
      \point {Created an emulator for a fictional CPU architecture}
      \point {Wrote and booted a simplistic proof-of-concept x86 bootloader}
  \hrulefill
  \section{Personal Projects}
    \experience{Mullvad Indicator}{2020}{Gnome3 shell extension}{Gnome Javascript}
      \point {Monitor and control connectivity to the Mullvad.net VPN service}
    \experience{com.fightcade.Fightcade}{2020}{Fightcade application Flatpak}{YAML/Shell}
      \point {Sandboxes a proprietary application for easy distribution on Linux and CloudReady}
      \point {Bundles custom 32-bit Wine build for running Windows emulators on Linux}
\end{main}
\end{document}